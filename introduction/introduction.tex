\graphicspath{{introduction/fig/}}

\chapter{Introduction}
\label{chap:introduction}

\section{Background}

Biosensors are defined as devices that measures biological and chemical reactions through the use of a physical transducer 
mechanism which in turn generates signals porportional to the concentration of an analyte in a sample.\cite{bhallaIntroductionBiosensors2016} This allows for the detection of various biological elements such as biomarkers which can be used to monitor health conditions or diagnose diseases.

Designing a low-cost easy to use device to take readings from biosensors, would thus provide a valuable tool for the early detection of diseases such as cancer. 
This would decrease the need for expensive labratory testing and allow for lower cost and more regular testing of patients using a point-of-care device.
Having a multiplexed device, able to take readings from multiple biosensors, would allow a single blood sample to be taken and multiple tests to be run on it with minimum involvement from a healthcare professional.

\section{Project Objectives}
This is a multifaceted project that aims to develop a low-cost, easy-to-use device for reading biosensors. The system design objectives for the device include:
\begin{itemize}
    \item Developing a circuit that allows a microcontroller to read a biosensor through the use of Electrochemical Impedance Spectroscopy (EIS)\index{Electrochemical Impedance Spectroscopy (EIS)}. 
    This entails conditioning the sinusoidal signal generated by a DAC and measuring the current response from the biosensor through an ADC.
    \item This measurement circuitry needs to be multiplexed to allow for multiple biosensors to be read in turn without the need for human intervention.
    \item The microcontroller should be able to filter and process the biosensor data to determine the concentration of the analyte in the sample.
    \item The device needs to be able to communicate these results in a clear and user-friendly manner. 
This will be done through the use of a LCD screen on the device as well as a web interface that can be accessed through a smartphone or computer.
    \item The device should be battery powered, low-cost and easy to use, thus allowing for widespread adoption and use in various environments.
\end{itemize}
\section{Project Scope}
NOT THE DESIGN OF Biosensors
NOT TESTING WITH REAL BLOOD (We arent qualified for that)

\section{Chapter Overview}

\textbf{\large Chapter 1: Introduction} \\
    Provides an overview of the motivation, background, objectives, scope, and structure of the report.
\\\\
\textbf{\large Chapter 2: Literature Review} \\
    Reviews relevant research and technologies in speech recognition and neural networks.
\\\\
\textbf{\large Chapter 3: Methodology} \\
    Describes the methods and approaches used in the project, including data collection and model development.
\\\\
\textbf{\large Chapter 4: Experiments and Results} \\
    Presents experimental setup, results, and analysis.
\\\\
\textbf{\large Chapter 5: Discussion} \\
    Discusses findings, limitations, and implications of the results.
\\\\
\textbf{\large Chapter 6: Conclusion and Future Work} \\
    Summarizes the work and suggests directions for future research.


