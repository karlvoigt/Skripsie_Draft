\graphicspath{{design/fig/}}

\chapter{Design and Simulation}\label{chap:design}
This chapter details the development of the BioPal impedance analyzer, progressing from user-centered design principles through circuit design, simulation, and firmware implementation. The design process was guided by understanding POC requirements and translating them into concrete technical specifications.

\section{Design Philosophy}

The design followed Brown's design thinking approach\cite{brownDesignThinking2008}, emphasizing user empathy and iterative prototyping. Considering the needs of \ac{POC} healthcare workers revealed that existing solutions, whether commercial instruments like the PalmSens4 or academic prototypes \cite{buscagliaSimpleZLowCostPortable2023}\cite{ al-aliDesignPortableLowCost2017}, share a critical limitation: they require users to interpret raw impedance spectra and understand EIS principles.

These insights shaped the project objectives around three core pillars: accessibility, affordability, and multiplexing and directly informed the functional architecture and component selection detailed in the following sections.

\section{Functional Design Overview}
The system can be broken up into multiple subsystems, each fulfilling a specific function in order to create a complete device that meets the requirements. On the most basic level the device consists of the \acp{DUT}, an impedance analyser and a user interface. The DUT is the \ac{IDE} that interacts with the analyte and whose impedance characteristics change based on the analyte concentration. A multiplexer is used to interface between multiple \acp{DUT} and the impedance analyser. The impedance analyser can be further broken down into the power supply, excitation stage, voltage and current measurement stages and finally the processing that uses an STM32. The user interface is based on an ESP32 and communicates with the impedance analyser through UART.

% \todo{Breek impedance analyser op in further subsystems}
% Discuss wat ons wil bereik en hoe ons dit gan opbreek. Beskryf hoe die circuit die beginsels van biosensors en EIS in ag moet neem, basies wat dit moet doen, hoekom multiplexing en ja dan hoe daai overall doel in gedeeltes opgebreek word.

\begin{figure}[H]
    \centering
    \includegraphics[width=0.8\textwidth]{FullSystemOverview.png}
    \caption{System Overview}
    \label{fig:system_overview} 
\end{figure}

The system architecture is shown in Figure \ref{fig:system_overview}. The ESP32 C6 has \ac{ADC} and \ac{DAC} peripherals, but their performance is known to be suboptimal for high-precision measurement applications. Furthermore, the lack of direct timer control or \ac{DMA} support for these peripherals limits their usability for synchronous signal generation and acquisition. The STM32F303K8 on the other hand is specifically designed for mixed-signal applications, with high-performance \ac{ADC} and \ac{DAC} peripherals, advanced timer capabilities and \ac{DMA} support. However, it lacks support for wireless communication and has limited flash and RAM resources, making it unsuitable for complex user interfaces. Thus, a dual-microcontroller architecture was used.

The following sections will first discuss the design of the analogue frontend and then the embedded firmware design.
\section{Analogue Frontend Design}
The analogue frontend is responsible for generating the excitation signal, measuring the voltage across and current through the \ac{DUT}, and conditioning these signals for digitization by the \ac{ADC} on the STM32F303K8. The design of each stage is driven by the requirements of accurate impedance measurement across a wide frequency range (1 Hz to 100 kHz) and dynamic range (from \SI{100}{\kilo\ohm} to \SI{10}{\ohm}). Each stage is discussed in detail below, including component selection, circuit design and theoretical performance analysis.

\subsection{Excitation Stage}\label{subsec:design_excitation}
The easiest way of producing a controlled voltage signal is using a \ac{DAC}. Both dedicated \acp{DAC} and \acp{DAC} built into \acp{MCU} were possible options. However the STM32F303K8 was chosen specifically due to its high-performance 12-bit \ac{DAC}. Using the internal \ac{DAC} simplifies the design by reducing component count and cost.

To avoid establishing a DC bias at the electrode–electrolyte interface, the generated excitation must be centred around a stable reference potential. This can be achieved by shifting the \ac{DAC} output to be biased around ground using a level-shifting op-amp circuit. However, this requires providing all analogue circuitry with a negative supply rail. This project instead uses a buffered virtual ground reference at 1.65 V (3.3 V/2) as the midpoint for all analogue circuitry. This approach ensures no DC bias is applied to the \ac{DUT} whilst negating the need for negative supply rails.

The LTC1069 was chosen as the AA-filtering \ac{IC}. It provides an 8th order lowpass filter that approximates a raised cosine response (with $\alpha=1$). It has a cutoff frequency of up to 120 kHz (200 kHz when using $\pm5$ V supply rails) set by an external clock and a linear phase response \cite{LTC10697CS8PBF}. The clock-tunable nature is ideal for this project, allowing easy adjustments through a timer on the STM.

\begin{equation}
    A_v = \frac{V_{out}}{V_{in}} = -\frac{R_f}{R_{in}}
    \label{eq:inv_opamp_gain}
\end{equation}

\begin{figure}[H]
    \centering
    \includegraphics[width=\textwidth]{ExcitationSchem.png}
    \caption{Complete Excitation Stage Circuit}
    \label{fig:excitation_stage_circuit}
\end{figure}

To maximise voltage resolution and minimise noise, the full linear range of the \ac{DAC} is utilised when generating the signal. This requires that the \ac{DAC} output signal is attenuated from 3 Vpp to the desired 10 mVpp using an inverting op-amp gain stage. From Equation \ref{eq:inv_opamp_gain}, $R_{f}=1 k\Omega$ and $R_{in}=300 k\Omega$ were calculated as suitable values.

The expected voltage levels across the \ac{DUT} are thus too small to be measured directly by the \ac{ADC} of the STM32F303K8 and a voltage measurement stage is required to amplify the signal to fully utilise the linear range of the \ac{ADC}. This is discussed in the following section.

\subsection{Voltage Measurement}
The STM32F303K8's \ac{ADC} has a theoretical range of 0-3.3V, however non-linearities and noise near the rails reduce the effective usable range. A gain of 300 V/V was thus chosen for the measurement stage to amplify the signal from 10 mVpp to 3 Vpp (0.1 V - 3.1 V). \Ac{GBW} represents the -3 dB bandwidth of an op-amp at unity gain. The -3 dB bandwidth of the op-amp can be calculated for a specific gain using Equation \ref{eq:GBW}. $A_{noise}$ represents the noise gain calculated in Equation \ref{eq:noise_gain}and accounts for non-ideal feedback effects and circuit imperfections \cite{fiore53GainBandwidthProduct2018}. This determines the magnitude and phase response of the amplifier at different frequencies, with higher gains resulting in lower bandwidths and thus more significant gain reductions and phase shifts.

\begin{equation}
    f_c = \frac{GBW}{A_{noise}}
    \label{eq:GBW}
\end{equation}
\begin{equation}
    A_{noise} = 1 + \frac{R_f}{R_{in}}
    \label{eq:noise_gain}
\end{equation}

Thus, amplifying the signal by a large factor in a single stage would introduce significant phase shifts and gain reductions at higher frequencies. To mitigate this, a two-stage amplification approach is employed.

The INA331 instrumentation amplifier was selected for the first stage of voltage measurement due to its combination of low offset voltage, high common-mode rejection and low input bias current. The INA331 features a typical offset voltage of 250 µV, which represents the inherent DC error between the input terminals when no differential signal is applied. It directly adds to the measured voltage, creating a systematic DC error that must be considered in calibration. The low input bias current of 0.5 pA avoids loading the \ac{DUT} and influencing current measurements. 

The device provides an internal gain of 5 V/V, configurable to higher gains through external resistors according to the relationship $G = 5 + 5\times \frac{R_2}{R_1}$. Choosing $R_1=1 k\Omega$ and $R_2=2 k\Omega$ results in a gain of 15 V/V. The bandwidth can be estimated from the datasheet to be 2.3 MHz (as seen in Figure \ref{fig:ina_bw}) at the chosen gain. Using equations \ref{eq:mag_rollof} and \ref{eq:phase_shift}, the gain reduction and phase shift at 100 kHz can be estimated as -0.004 dB and -2.49\textdegree\ respectively. While this still needs to be accounted for during calibration, it represents a very flat and linear response leading to a more accurate system.


\begin{equation}
    |H(j\omega)|_{dB} = 20\log\frac{1}{\sqrt{1+\frac{\omega^2}{w_0^2}}}
    \label{eq:mag_rollof}
\end{equation}
\begin{equation}
    \varphi(\omega) = -\tan^{-1}(\frac{\omega}{\omega_0})
    \label{eq:phase_shift}
\end{equation}

% \todo{Move to appendix}
% \begin{figure}[H]
%     \centering
%     \includegraphics[width=0.5\textwidth]{INA_BW.jpeg}
%     \caption[INA331 Bandwidth vs Gain]{INA331 Bandwidth vs Gain adapted from \cite{INA331}}
%     \label{fig:ina_bw}
% \end{figure}
% For standard inverting and non-inverting op-amp gain topologies, the gain and phase shift at any frequency can be calculated based on the cutoff frequency using equations \ref{eq:mag_rollof} and \ref{eq:phase_shift} respectively \cite{oljacaOperationalAmplifierGain2010}, with $\omega=2\pi f$ and $\omega_0=2\pi f_c$.

The second stage uses a TLV9061 op-amp in an inverting gain configuration.  With a GBW of 10 MHz and gain of $A_v=-20$ ($A_{noise}=21$), the expected bandwidth for the TLV gain stage is 476.2 kHz (Equation \ref{eq:GBW}). From equations \ref{eq:mag_rollof} and \ref{eq:phase_shift}, an expected -0.094 dB gain reduction and -11.86\textdegree\ phase shift is calculated at 100 kHz. However, this does not need to be calibrated for as an identical gain stage will be used for the current measurement, ensuring that any gain reductions and phase shifts are cancelled out during impedance calculation.

The final circuit for voltage measurement can be seen in Figure \ref{fig:vmeas_stage_circuit}.

\begin{figure}[H]
    \centering
    \includegraphics[width=\textwidth]{VMeasSchem.png}
    \caption{Complete Voltage Measurement Stage Circuit}
    \label{fig:vmeas_stage_circuit}
\end{figure}

\todo{Add linking paragraph}

\subsection{Current Measurement}\label{subsec:design_cur}
The current measurement stage represents the most complex and arguably most important aspect of the analogue frontend. Accurate current measurement across a wide dynamic range is essential for reliable impedance determination, thus significant effort was spent on the design of this stage.

The architecture consists of three stages: a \ac{TIA} provides the initial current-to-voltage conversion, followed by a \ac{PGA}, then a final inverting gain op-amp stage. This final stage is identical to the second voltage measurement stage, ensuring that gain reductions and phase shifts cancel during impedance calculation.

Based on measurements of the biosensor using the PalmSens4, the expected impedance ranges from \SI{100}{\kilo\ohm} at 1 Hz to \SI{10}{\ohm} at 100 kHz. For a fixed 10 mVpp excitation, this corresponds to a current range from 100 nA to 1 mA, spanning four orders of magnitude. This wide dynamic range drives several critical design requirements.

A fixed-gain amplifier is impractical across this range. High gain would saturate at high currents, whilst low gain would not utilise the full \ac{ADC} range at low currents. Programmable gain is therefore essential. This is achieved through two mechanisms: switchable feedback resistors in the \ac{TIA} and variable gain in the \ac{PGA} stage.

The PGA113 offers gains ranging from 1-200 V/V with a high GBW of 10 MHz and is controlled via SPI. It has a low gain error of $\le0.3\%$ and extremely low noise at $12\text{ nV}/\sqrt{\text{Hz}}$. Combined with the final TLV9061 gain stage (identical to the final voltage measurement stage), this provides variable gain from 20 V/V - 4000 V/V after the \ac{TIA}. 

% While the \ac{TIA}'s feedback resistor can be large without affecting the applied signal to the \ac{DUT}, it reduces the bandwidth. If $R_{feedback}$ is too large, the \ac{TIA} experiences significant phase shifts and reduced gain at higher frequencies (referring back to Equation \ref{eq:GBW}). Using multiple gain stages distributes the amplification, allowing the \ac{TIA} to use a moderate feedback resistor whilst maintaining adequate bandwidth.

Multiple feedback resistor paths on the \ac{TIA} enable finer gain segmentation across the input current range, improving measurement precision. The larger feedback resistor handles low currents with maximum resolution, whilst the smaller resistor prevents saturation at high currents. 

The OPA3S328 is specifically designed for \ac{TIA} applications, with a wide GBW of 40 MHz, 0.2 pA input bias and typical input voltage offset of \SI{10}{\micro\volt}. Importantly, it has integrated switches for switching between feedback resistors. However, the switch on-resistance is non-negligible at 90--125 $\Omega$ and varies with temperature. This produces gain errors and distortion on the \ac{TIA} output. This can be addressed by using the second switch and op-amp integrated into the OPA3S328 package to build a buffered multiplexer. An example of this circuit is shown in Figure \ref{fig:kelvin_sense_tia}. The switch senses the \ac{TIA} output directly at the feedback resistor for each gain, whilst the second op-amp acts as a buffer. The low input bias current of the op-amp ensures negligible voltage drop (a worst case of 1.25 nV), providing an accurate Kelvin sense connection.

% \todo{Move to appendix}
% \begin{figure}[]
%     \centering
%     \includegraphics[width=0.8\textwidth]{KelvinSenseTIA.png}
%     \caption[Example of Kelvin Sense Switch Connections and High Impedance Buffer]{Example of Kelvin Sense Switch Connections and High Impedance Buffer\cite{chioyeBuildProgrammableGain2021}}
%     \label{fig:kelvin_sense_tia}
% \end{figure}

% \todo{remove this:}
% The YBJ variant was chosen due to its 3-way \ac{MUX} over the RGR variant with a 2-way \ac{MUX}, enabling finer gain segmentation and improved measurement precision across the input current range. However, due to the small package size and PCB manufacturing limits (see section \ref{sec:PCB}), only two of the three switches were used.

To maximise the \ac{ADC}'s range even at the smallest current, the larger feedback resistor is designed to deliver 3 Vpp at the maximum \ac{PGA} gain. From equations \ref{eq:tia_1}--\ref{eq:tia_3}, this results in 7.5 k$\Omega$.
\begin{align}
        V_{TIA} &= \frac{3}{4000} = 750\ \mu V \label{eq:tia_1}\\
        A_{TIA} &= \frac{750\ \mu V}{100\ nA} = 7500\ V/A \\
        \therefore  R_{f1} &= 7.5\ k\Omega \label{eq:tia_3}
\end{align}

The smaller feedback resistor is chosen to give the same overall gain at the maximum \ac{PGA} gain as the larger resistor gives at the minimum \ac{PGA} gain. This ensures smooth transitions between gain settings without gaps where currents would be too large for one resistor but too small for the other. The smaller feedback resistor is calculated to be 200 times smaller at $R_{f2}=37.5 \Omega$. This ensures that the ranges from 50 nA - 20 $\mu$A and 20 $\mu$A - 4 mA are covered respectively, whilst ensuring that the \ac{ADC} input stays between 1.5 Vpp and 3 Vpp.

In traditional TIA designs a capacitor is placed in parallel with the feedback resistor to provide sufficient phase margin and ensure stability \cite{StabilizeYourTransimpedance}. These designs are, however intended for use with a photodiode, rather than \ac{EIS}. There are few sources available that discuss the design of a TIA for EIS purposes, where capacitance is what is measured rather than compensated for. These sources do not discuss the necessity of feedback capacitors, thus this has to be derived from theory.

% The dominant pole of the open loop response of an op-amp is a function of the GBW and the open loop gain, $A_{OL}$ (Equation \ref{eq:fdp}). This determines the -3dB frequency of the open loop transfer function and after this point the gain rolls of at a rate of -20dB/decade. 
The amount of the output that is fed back to the input of an op-amp is defined as the feedback factor ($\beta$) and described by Equation \ref{eq:fb_factor}. Where $V_{fb}$ is the voltage present at the inverting input of the op-amp and $V_{out}$ the voltage at the output. This negative feedback results in the closed loop gain (Equation \ref{eq:acl}) where the loop-gain refers to $A_{OL}\times\beta$ (Equation \ref{eq:loop_gain}).
\begin{align}
    % \text{Dominant Pole} &= f_{DP} = \frac{\text{GBW}}{A_{OL}} \label{eq:fdp}\\
    \text{Feedback Factor} &= \beta = \frac{V_{fb}}{V_{out}}  \label{eq:fb_factor}\\
    \text{Closed Loop Gain} &= A_{CL} = \frac{A_{OL}}{1 + A_{OL}\beta} \label{eq:acl}\\
    \text{Loop Gain} &= A_{OL}\beta \label{eq:loop_gain}\\
    \text{PM} &= 180^\circ + \varphi_{A_{OL}\beta} &\text{ at } |A_{OL}\beta| = 0dB \label{eq:pm} 
\end{align}
When $A_{OL}\beta = -1$ the system becomes unstable. This happens when $V_{fb}$ leads or lags $V_{out}$ by 180\textdegree. The phase margin (PM) is thus defined as in Equation \ref{eq:pm} and describes how close the system is to a 180\textdegree\ phase shift and instability when the loop gain magnitude is at 0dB or 1V/V (the critical point or $f_c$). To ensure stability, the PM should be at least 45\textdegree, with higher values providing more stability at the cost of transient response \cite{StabilizeYourTransimpedance}. 
% The phase margin can be calculated by subtracting the phase of $\frac{1}{\beta(j\omega)}$ from the phase of $A_{OL}(j\omega)$ to get the phase of $A_{OL}\beta(j\omega)$.
% From Equation \ref{eq:aol_at_0db} it is shown that this happens when the magnitude plots of $A_{OL}$ and $\frac{1}{\beta}$ intersect.
% \begin{align}
    % \text{PM} &= 180^\circ + \varphi_{A_{OL}\beta} \text{ at } |A_{OL}\beta| = 0dB \label{eq:pm} \\
    % |A_{OL}\beta| &= 0dB = 1 V/V\\
    % \therefore |A_{OL}| &= |\frac{1}{\beta}| \text{ at } f_c \label{eq:aol_at_0db}
% \end{align}
% \begin{figure}[H]
%     \centering
%     \begin{minipage}{0.6\textwidth}
%         \centering
%         \includegraphics[width=\textwidth]{PhotoDiodeCircuit.png}
%         \caption[Photodiode Equivalent Circuit]{Photodiode Equivalent Circuit\cite{StabilizeYourTransimpedance}}
%         \label{fig:photodiode}
%     \end{minipage}\hfill
%     \begin{minipage}{0.4\textwidth}
%         \centering
%         \includegraphics[width=\textwidth]{PhotoTIA.png}
%         \caption{Equivalent TIA Circuit for Photodiodes}
%         \label{fig:tia_photo}
%     \end{minipage}
% \end{figure}
\begin{align}
    C_i &= C_j+C_{in} \\
    \beta (j\omega) &= \frac{X_{c_i}}{R_f + X_{C_i}} = \frac{1}{1+j\omega R_fC_i} \label{eq:photo_beta}
    % \therefore \frac{1}{\beta(j\omega)} &= 1+j\omega R_fC_i
\end{align}
The simplified equivalent circuit of a photodiode TIA circuit has the photodiode junction capacitance ($C_J$) in parallel with the input capacitance of the op-amp ($C_{in}$) (see Figures \ref{fig:photodiode} and \ref{fig:tia_photo} in Appendix \ref{append:design}). The feedback factor can be calculated as seen in Equation \ref{eq:photo_beta}. The pole in $\beta(j\omega)$, caused by the combined input capacitance, causes the magnitude of $\beta(j\omega)$ to decrease at a rate of -20dB/decade after the corner frequency (determined by the value of $C_{i}$) and the phase to decrease from 0\textdegree to -90\textdegree. The result is a very small phase margin at the critical frequency and instability (as seen in Figure \ref{fig:matlab_lg}). The feedback capacitor in parallel with $R_f$ solves this by adding a zero to $\beta(j\omega)$, cancelling out the pole and thus adding phase margin.

%Figures \ref{fig:randles_cpe_tia} and \ref{fig:randles_cdl_tia} show the equivalent TIA circuits for a non-faradaic Randles circuit and a simplified model using an ideal capacitance ($C_{dl}$) instead of the \ac{CPE} by approximating $T=C_{dl}$ and $\alpha=1$.
However, the equivalent Randles circuit model of a biosensor differs fundamentally from the model of a photodiode due to the series resistance $R_s$. The input impedance and feedback factor for the Randles non-faradaic equivalent circuit can be calculated as:
\begin{figure}[H]
    \centering
    \begin{minipage}[t]{0.48\textwidth}
        \vspace{0pt}
        % Ideal Capacitance:
        % \begin{align}
        %     Z_{in} &= (R_S + X_{C_{DL}}) || X_{C_{in}} \label{eq:simple_zin}\\
        %     \beta (j\omega) &= \frac{Z_{in}}{R_f + Z_{in}} \label{eq:simple_beta} \\
        %     % \therefore \frac{1}{\beta(j\omega)} &= 1 + \frac{R_f}{Z_{in}}\\
        %     % &= 1 + \frac{R_f}{(R_S + X_{C_{DL}}) || X_{C_{in}}} \nonumber
        % \end{align}
        \centering
        \includegraphics[width=\textwidth]{RandlesTIA.png}
        \caption{Randles Equivalent TIA Circuit with CPE}
        \label{fig:randles_cpe_tia}
    \end{minipage}\hfill
    \begin{minipage}[t]{0.48\textwidth}
        \vspace{0pt}
        \begin{align}
            Z_{CPE} &= \frac{1}{T(j\omega)^{\alpha}} \\
            Z_{in} &= (R_S + Z_{CPE}) || X_{C_{in}} \\
            \beta (j\omega) &= \frac{Z_{in}}{R_f + Z_{in}} \label{eq:randles_beta}\\
            % \therefore \frac{1}{\beta(j\omega)} &= 1 + \frac{R_f}{Z_{in}} \\
            % &= 1 + \frac{R_f}{(R_S + Z_{CPE}) || X_{C_{in}}} \nonumber
        \end{align}
    \end{minipage}
\end{figure}

% \begin{figure}[H]
%     \centering
%     \begin{minipage}{0.5\textwidth}
%         \centering
%         \includegraphics[width=\textwidth]{RandlesSimpleTIA.png}
%         \caption{Simplified Randles Equivalent TIA Circuit with Ideal Capacitance}
%         \label{fig:randles_cdl_tia}
%     \end{minipage}\hfill
%     \begin{minipage}{0.5\textwidth}
%         \centering
%         \includegraphics[width=\textwidth]{RandlesTIA.png}
%         \caption{Randles Equivalent TIA Circuit with CPE}
%         \label{fig:randles_cpe_tia}
%     \end{minipage}
% \end{figure}

The circuit parameters of the IDE were calculated using circuit fitting in the PSTrace software that accompanies the PalmSens (see Table \ref{tab:stability_params}) and values for input capacitance, open loop gain and gain bandwidth were read from the OPA3S328 datasheet. A MatLab script was then used to calculate the loop gain of the system.  The response of a Randles equivalent non-fardaic circuit with constant $C_{dl}$ instead of a CPE and the equivalent photodiode model were also plotted. The resulting loop-gain responses are shown in Figure \ref{fig:matlab_lg} and Table \ref{tab:matlab_results} lists the calculated critical frequencies and phase margins.

\begin{figure}[H]
    \centering
    % \begin{subfigure}[b]{0.5\textwidth}
    %     \centering
    %     \includegraphics[width=\textwidth]{MatLabLG_37,5.png}
    %     \caption{$R_F=37.5\Omega$}
    %     \label{fig:matlab_cl_37.5}
    % \end{subfigure}\hfill
    % \begin{subfigure}[b]{0.5\textwidth}
        \centering
        \includegraphics[width=\textwidth]{MatLabLG_7500.png}
        % \caption{$R_F=7.5k\Omega$}
        % \label{fig:matlab_cl_7.5}
    % \end{subfigure}
    \caption{Loop Gain Frequency Response for $R_F=7.5k\Omega$}
    \label{fig:matlab_lg}
\end{figure}
% \todo{Move this to the appendix}
% \begin{figure}[H]
%     \centering
%     \begin{subfigure}[b]{0.5\textwidth}
%         \centering
%         \includegraphics[width=\textwidth]{MatLabBeta_37,5.png}
%         \caption{$R_F=37.5\Omega$}
%         \label{fig:matlab_beta_37.5}
%     \end{subfigure}\hfill
%     \begin{subfigure}[b]{0.5\textwidth}
%         \centering
%         \includegraphics[width=\textwidth]{MatLabBeta_7500.png}
%         \caption{$R_F=7.5k\Omega$}
%         \label{fig:matlab_beta_7.5}
%     \end{subfigure}
%     \caption{$\frac{1}{\beta(j\omega)}$ Frequency Responses}
%     \label{fig:matlab_beta}
% \end{figure}
At low frequencies the IDE circuit closely follows the response of the photodiode. This is due to the zero caused by the capacitance dominating the response ($X_{C_{DL}}>>R_S$). However, at higher frequencies $R_S$ dominates and the responses start to diverge. This confirms that ample phase margin is available at both $R_f=37.5\Omega$ and $R_f=7.5k\Omega$, meaning that no feedback capacitors are needed. At extremely large values of $R_F$ the phase margin reduces, and the system can become unstable ($R_f>40k\Omega$ for this circuit). Since adding compensation capacitors reduces the bandwidth of the TIA \cite{StabilizeYourTransimpedance}, it is not recommended for biosensing TIA circuits, except in cases of extremely high feedback resistor values.
\begin{table}[H]
    \centering
    \begin{minipage}{0.3\textwidth}
        \centering
        \caption{Circuit Parameters}
        \label{tab:stability_params}
        \begin{tabular}{ll}
            \hline
            \textbf{Parameter} & \textbf{Value} \\
            \hline
            \multicolumn{2}{l}{\textit{CPE Randles Circuit}} \\
            $R_S$ & \SI{8.975}{\ohm} \\
            $T$ & \SI{6.993e-6}{} \\
            $\alpha$ & 0.785 \\
            $C_{in}$ & \SI{4.0}{\pico\farad} \\
            \hline
            \multicolumn{2}{l}{\textit{Constant C Randles Circuit}} \\
            $R_S$ & \SI{12.3}{\ohm} \\
            $C_{DL}$ & \SI{1436.0}{\nano\farad} \\
            $C_{in}$ & \SI{4.0}{\pico\farad} \\
            \hline
            \multicolumn{2}{l}{\textit{Photodiode Equivalent}} \\
            $C_i$ & \SI{1436.0}{\nano\farad} \\
            \hline
        \end{tabular}
    \end{minipage}\hfill
    \begin{minipage}{0.6\textwidth}
        \centering
        \caption{Stability Analysis Results}
        \label{tab:matlab_results}
        \begin{tabular}{lccc}
            \hline
            \textbf{Configuration} & $f_c$ & \textbf{PM} & \textbf{Status} \\
             & \textbf{(kHz)} & \textbf{(\textdegree)} & \\
            \hline
            \multicolumn{4}{l}{\textit{$R_f = \SI{7.5}{\kilo\ohm}$}} \\
            Constant C Randles & 65.83 & 82.2 & Stable \\
            Photodiode & 24.22 & 0.1 & Unstable \\
            CPE Randles & 64.22 & 63.8 & Stable \\
            \hline
            \multicolumn{4}{l}{\textit{$R_f = \SI{37.5}{\ohm}$}} \\
            Constant C Randles & 9840.03 & 89.8 & Stable \\
            Photodiode & 343.77 & 0.5 & Unstable \\
            CPE Randles & 7725.81 & 86.3 & Stable \\
            \hline
        \end{tabular}
    \end{minipage}
\end{table}

The final current measurement circuit is shown in Figure \ref{fig:imeas_stage_circuit}.
\begin{figure}[H]
    \centering
    \includegraphics[width=\textwidth]{IMeasSchem.png}
    \caption{Complete Current Measurement Stage Circuit}
    \label{fig:imeas_stage_circuit}
\end{figure}
\todo{Insert Linking paragraph}

\subsection{Multiplexer and IDE interface}
While the design and manufacture of \acp{IDE} are outside the scope of this project. The \acp{IDE} described in \cite{ebrahimDevelopmentBiosensorEarly2023} were used for this project. To ensure ease-of-use, a method of interfacing with the biosensors that is simple and reliable needed to be developed. Spring-loaded battery connectors were used as they allow the DUT to be easily slid in and out of the device when combined with a 3d printed housing.
\todo{Insert 3d model of connectors, dut and 3d print.}

In order to allow multiple \acp{IDE} to be measured by a single analogue front end, a multiplexing solution was required. Various options for multiplexers were considered including dedicated analogue multiplexers (MUX ICs) and relays. Dedicated analogue multiplexers consist of a collection of analogue switches. They typically use CMOS technology, resulting in compact integration and fast switching speeds. Modern analogue switches are available with very low on-resistance ($<1\Omega$) and a high degree of flatness \cite{SelectingRightCMOS}. However, leakage currents are inherent to these solid-state devices and can corrupt low-current signals, especially in the nanoampere range \cite{SelectingRightCMOS}. 

% Op-amp-based multiplexers such as seen in Figure \ref{fig:opamp_mux} provide buffering and impedance matching, which make them ideal for multiplexing voltage signals, but the buffering also make them unsuitable for use with current signals.

% \begin{figure}[H]
%     \centering
%     \includegraphics[width=0.4\textwidth]{OpAmpMux.png}
%     \caption[Op-amp Based Multiplexer Circuit]{Op-amp Based Multiplexer Circuit \cite{Sboa311a}}
%     \label{fig:opamp_mux}
% \end{figure}

Signal relays, in contrast, use electromechanical contacts to physically open or close signal paths, offering near-zero leakage current and extremely low, stable contact resistance that is independent of signal voltage and temperature. This physical isolation and connection ensures that the measured current accurately reflects the biosensor response. While relays are slower to switch and larger than solid-state alternatives, their switching speed is more than sufficient for switching between sensors.
\begin{figure}[H]
    \centering
    \begin{minipage}{0.35\textwidth}
        \centering
        \includegraphics[width=\textwidth]{RelayTopologySchem.png}
        \caption[Relay Multiplexer Topology for 4 DUT's]{\newline Relay Multiplexer Topology for 4 \acp{IDE}}
        \label{fig:relay_topology}
    \end{minipage}\hfill
    \begin{minipage}{0.6\textwidth}
        \centering
        \includegraphics[width=\textwidth]{RelayDriverSchem.png}
        \caption{Relay Driver Circuit for one relay}
        \label{fig:relay_circuit}
    \end{minipage}
\end{figure}
The TXS2-L2-3V DPDT latching signal relay was chosen due to its small size, low operating current (23.3 mA) and high mechanical lifetime (Minimum 200 000 operations). Utilising the DPDT topology of the relay, they can be configured in a tree pattern, allowing for 4 \acp{IDE} to be switched using 3 relays as seen in Figure \ref{fig:relay_topology}. 

Despite the low operating current, a driver circuit is still needed to power the relay from a microcontroller's \ac{GPIO} pins. This consists of a lowside NPN transistor and a flyback diode to protect against voltage spikes when the coil is switched off. The final circuit can be seen in Figure \ref{fig:relay_circuit}

Finally, a power supply system needed to be designed to power the entire system.
\subsection{Power Supply}
To ensure portablility, a battery is required. The most cost-effective approach is to utilise a microcontroller with built-in LiPo charging circuitry instead of a dedicated charging circuit. The Firebeetle 2 ESP32 C6 uses an HM6245 \ac{LDO}, which can supply up to 1A. This is sufficient to power the rest of the system, from the 3.3V supply of the ESP32 dev board.

As mentioned in Section \ref{subsec:design_excitation}, a 1.65V reference is needed for the analogue circuitry. Due to the small amplitude of the excitation signal, it needs to be highly accurate and stable. This was done through the use of a matched resistor array and an op-amp buffer. Using a resistor array ensures that our reference is the exact midpoint of the supply voltage despite any tolerances in the resistor value, while the op-amp buffers this output to avoid loading the resistor array and causing a voltage drop. Choosing a too large resistor value risks a slightly uneven voltage drop due to the small input bias current of the op-amp buffer, on the other hand, a lower value increases the static current draw and power consumption. $1k\Omega$ was chosen as a balance between these trade-offs.

\begin{figure}[H]
    \centering
    \begin{minipage}{0.35\textwidth}
        \centering
        \includegraphics[width=\textwidth]{Vground.png}
        \caption[Virtual Ground Reference Circuit]{\newline Virtual Ground Reference Circuit}
        \label{fig:virtual_ground}
    \end{minipage}\hfill
    \begin{minipage}{0.6\textwidth}
        \centering
        \includegraphics[width=\textwidth]{5V_Reg.png}
        \caption{5V Boost Converter Circuit}
        \label{fig:5V_reg}
    \end{minipage}
\end{figure}

The LTC1069 AA-Filter requires a 5V supply voltage. A 3.3V to 5V boost circuit was thus designed around the TPS61072 boost regulator using the TI WeBench power supply design tool \cite{WEBENCHCIRCUITDESIGNERDesignTool}, ensuring a stable and efficient circuit as seen in Figure \ref{fig:5V_reg}. 

With the complete analogue frontend designed, the next step was to simulate the entire circuit to verify functionality before moving on to PCB design.

\section{Circuit Simulation}
All circuit simulations were performed in LTSpice XVII. The overall system was broken down into subsystems for ease of simulation and verification. Each subsystem was simulated separately before combining them into a full system simulation. All component SPICE models were obtained from the respective manufacturer websites, however for TI products, this required converting PSpice models to LTSpice format.

An accurate model of the \ac{IDE} is essential for meaningful simulation results and thus was the first step in the simulation process.
\subsection{IDE}
As discussed in Section \ref{subsec:lit_review_eis_sensors}, \acp{CPE} are commonly used to model \ac{IDE} biosensors due to their non-ideal capacitive behaviour. Despite the widespread use of \acp{CPE} in electrical simulations, the SPICE family of simulators lack a native CPE element. The approach described in \cite{wilsonSimulatingFractionalCapacitors2023} models a CPE using an array of parallel RC elements. The branches form a theoretically infinite geometric progression of characteristic frequencies \cite{wilsonSimulatingFractionalCapacitors2023}, however characteristic frequencies above and below the frequency range of interest are approximated using a single capacitor and resistor respectively. The provided MatLab code was used to calculate the R and C values of all the branches and create a LTSpice model of the CPE based on the paramaters obtained from the PSTrace (Table \ref{tab:stability_params}). The resulting IDE model closely matches the frequency response obtained from PSTrace (Figure \ref{fig:ide_freq} in Appendix \ref{appen:simulations}). With the model of the \ac{IDE} confirmed to be accurate, the rest of the system could be simulated.
% \todo{Move these to appendix}
% \begin{figure}[H]
%         \centering
%         \includegraphics[width=\textwidth]{CPE_linear.png}
%         \caption{Simulated CPE Frequency Response}
%         \label{fig:cpe_sim}
% \end{figure}
% \begin{figure}[H]
%         \centering
%         \includegraphics[width=\textwidth]{DUT_Z.png}
%         \caption{Simulated Biosensor Frequency Response}
%         \label{fig:biosensor_sim}
% \end{figure}

\subsection{Excitation Stage}
For the simulation of the excitation stage, a sample-and-hold block was used to mimic the DAC at varying sample rates. LTSpice has no included components to simulate a raised cosine filter, thus a FIR filter block from \cite{FilterManual} with a raised cosine response and $\beta = 1$ was used. The simulated frequency response compared well with the graphs provided in the LTC1069-7 datasheet (Figures \ref{fig:lt_aa_freq} and \ref{fig:ltc_aa_freq} in Appendix \ref{appen:simulation}), despite some differences at high levels of attenuation ($<-50dB$). The TLV9061 model and calculated resistor values were used for the simulation of the attenuation stage and the \ac{IDE} model was connected to complete the excitation stage simulation.

% \todo{Move these to appendix}
% \begin{figure}[H]
%     \centering
%     \begin{subfigure}[b]{0.48\textwidth}
%         \centering
%         \includegraphics[width=\textwidth]{FIR_10k.png}
%         \caption{10kHz Cutoff Frequency}
%         \label{fig:fir_10k}
%     \end{subfigure}\hfill
%     \begin{subfigure}[b]{0.48\textwidth}
%         \centering
%         \includegraphics[width=\textwidth]{FIR_100k.png}
%         \caption{100kHz Cutoff Frequency}
%         \label{fig:fir_100k}
%     \end{subfigure}
%     \caption{Simulated FIR filter Frequency Responses}
%     \label{fig:lt_aa_freq}
% \end{figure}

% \begin{figure}[H]
%     \centering
%     \begin{subfigure}[b]{0.4\textwidth}
%         \centering
%         \includegraphics[width=\textwidth]{LTC_10k.png}
%         \caption{10kHz Cutoff Frequency}
%         \label{fig:ltc_10k}
%     \end{subfigure}\hfill
%     \begin{subfigure}[b]{0.4\textwidth}
%         \centering
%         \includegraphics[width=\textwidth]{LTC_100k.png}
%         \caption{100kHz Cutoff Frequency}
%         \label{fig:ltc_100k}
%     \end{subfigure}
%     \caption{LTC1069-7 Datasheet Frequency Responses}
%     \label{fig:ltc_aa_freq}
% \end{figure}
% \begin{figure}[H]
%     \centering
%     \begin{subfigure}[b]{0.48\textwidth}
%         \centering
%         \includegraphics[width=\textwidth]{FullExcitationStage.png}
%         \caption{Time Domain Response}
%         \label{fig:dac_filtering}
%     \end{subfigure}\hfill
%     \begin{subfigure}[b]{0.48\textwidth}
%         \centering
%         \includegraphics[width=\textwidth]{AA_vs_noAA_cur.png}
%         \caption{Frequency Domain Response}
%         \label{fig:dut_current}
%     \end{subfigure}
%     \caption{Simulated Excitation Stage}
%     \label{fig:excitation_sim}
% \end{figure}
\begin{figure}[H]
    \centering
    \includegraphics[width=0.7\textwidth]{FullExcitationStage.png}
    \caption{Simulated Excitation Stage}
    \label{fig:excitation_sim}
\end{figure}

Figure \ref{fig:excitation_sim} shows a 10kHz signal with 32x oversampling generated by the DAC before and after passing through the AA-filter and after being attenuated. The circuit performed as expected and the voltage measurement stage was simulated next.

\subsection{Voltage Measurement}
The voltage measurement stage was simulated with the INA331 and TLV9061 models to confirm the frequency response of the system. Figure \ref{fig:ina_freq} shows that the INA331 circuit exhibits only a slight gain reduction of -0.02dB and a -4.6\textdegree phase shift at 100kHz. Fig \ref{fig:v_meas_freq} shows the overall frequency response of the voltage measurement stage with the TLV9061 amplification stage contributing an additional -0.2dB gain reduction and -3.4\textdegree phase shift. This confirms that sufficient bandwidth is available for measurements up to 100kHz.

\begin{figure}[H]
    \centering
    \begin{minipage}{0.48\textwidth}
        \centering
        \includegraphics[width=\textwidth]{INA331FreqResponse.png}
        \caption{INA331 Frequency Response}
        \label{fig:ina_freq}
    \end{minipage}\hfill
    \begin{minipage}{0.48\textwidth}
        \centering
        \includegraphics[width=\textwidth]{VmeasFreqResponse.png}
        \caption{Complete Voltage Measurement Stage Frequency Response}
        \label{fig:v_meas_freq}
    \end{minipage}
\end{figure}

\subsection{Current Measurement}
Due to problems porting the OPA3s328 PSpice model to LT Spice, the internal mux was not simulated. The PGA113 has no available PSpice model \cite{PGA113PspiceModel2022}, thus a standard inverting amplifier configuration using the TLV9061 was used to simulate the PGA stage. Figure \ref{fig:tia_freq} shows the frequency response of the TIA stage alone at both feedback resistor values. At 100 kHz with $R_F = 7.5~k\Omega$, the stage exhibits a gain increase of +0.136 dB and a phase shift of -15.4\textdegree. For $R_F = 37.5~\Omega$, the response shows a gain increase of +0.137 dB and a phase shift of -15.6\textdegree. The slight gain increase is due to very slight peaking in the frequency response before the eventual rolloff. This demonstrates a very flat frequency response well below the circuit's bandwidth limitations. Figure \ref{fig:i_meas_freq} shows the overall frequency response of the current measurement stage with the PGA and final amplification stages contributing an additional -0.25dB gain reduction and -4.8\textdegree phase shift. This confirms that sufficient bandwidth is available for measurements up to 100kHz.

To confirm the stability of the TIA circuit, Tian's method was used to plot the loop gain and phase margin for both the Randles circuit using the \ac{CPE} and the simplified circuit using a constant capacitance as seen in Figure \ref{fig:tia_stability} and Table \ref{tab:lt_tia_stability}. These results closely match the MatLab results in section \ref{subsec:design_cur}, confirming that the TIA design is stable without the need for feedback compensation.

% \todo{move to appendix}
% Two sub figures for two tia gain freq responses using subfigures not minipages
% \begin{figure}[H]
%     \centering
%     \begin{subfigure}[b]{0.48\textwidth}
%         \centering
%         \includegraphics[width=\textwidth]{TIA_Freq_37,5.png}
%         \caption{$R_F = 37.5~\Omega$}
%         \label{fig:tia_freq_37.5}
%     \end{subfigure}\hfill
%     \begin{subfigure}[b]{0.48\textwidth}
%         \centering
%         \includegraphics[width=\textwidth]{TIA_Freq_7500.png}
%         \caption{$R_F = 7.5~k\Omega$}
%         \label{fig:tia_freq_7.5k}
%     \end{subfigure}
%     \caption{Simulated TIA Frequency Response}
%     \label{fig:tia_freq}
% \end{figure}

% Two subfigures for the complete current measurement stage freq response using subfigures not minipages for the two tia gain values
\begin{figure}[H]
    \centering
    \begin{subfigure}[b]{0.48\textwidth}
        \centering
        \includegraphics[width=\textwidth]{CurMeas_37.5.png}
        \caption{$R_F = 37.5~\Omega$}
        \label{fig:i_meas_freq_37.5}
    \end{subfigure}\hfill
    \begin{subfigure}[b]{0.48\textwidth}
        \centering
        \includegraphics[width=\textwidth]{CurMeas_7500.png} 
        \caption{$R_F = 7.5~k\Omega$}
        \label{fig:i_meas_freq_7.5k}
    \end{subfigure}
    \caption{Simulated Complete Current Measurement Stage Frequency Response}
    \label{fig:i_meas_freq}
\end{figure}

\subsection{Complete System}
The individual subsystems were combined into a complete model including DAC and ADCs using sample-and-hold blocks. Transient analysis was done at a range of excitation frequencies. For frequency analysis the DAC and ADCs were excluded.

\section{PCB Design}\label{sec:PCB}
% Beskryf filosofie en idees wat mee ingegaa het. Briefly discuss general PCB design principles wat design geguide het (Analogue ground plane etc). Beskryf beperkings van PCB manufactures wat inag geneem moes word (PCB size, layers, trace width via diamtre etc.). Noem briefly hoekom PCB in China eerder as Uni laat maak. Gaan deur design logic en discuss probleem met TIA. Include maybe final PCB diagram en langs dit foto van manufactured PCB.

All PCB design was done using KiCad due to its open-source nature and wide usage in industry. Due to the complexity of the circuit, JLC PCB was used for manufacturing rather than Stellenbosch University's in-house PCB manufacturing. \rephrase{There are substantial price differences between JLC's standard process and their more advanced processes (\textdollar 4 vs \textdollar 68)}. Table \ref{tab:jlc_limits} lists the key limits of JLC's standard PCB process that had to be taken into account. 

\begin{figure}[H]
    \centering
    \begin{subfigure}[b]{0.48\textwidth}
        \centering
        \includegraphics[width=\textwidth]{BioPal_Front.png}
        \caption{PCB front layout}
        \label{fig:pcb_front}
    \end{subfigure}\hfill
    \begin{subfigure}[b]{0.48\textwidth}
        \centering
        \includegraphics[width=\textwidth]{BioPal_Back.png}
        \caption{PCB back layout}
        \label{fig:pcb_back}
    \end{subfigure}
    \caption{Final PCB Layout}
    \label{fig:pcb_layout}
\end{figure}

Another key consideration when designing the PCB was minimising noise and interference in the analogue circuitry. Given the limitations of a two-layer PCB, maintaining a continuous and low-impedance ground reference was a key priority, as a fully dedicated ground plane was not practical. To achieve this, both layers incorporated extensive ground copper pours connected through frequent stitching vias to minimise loop inductance and reduce \ac{EMI} coupling between layers. A single unified ground network was chosen over separate analogue and digital grounds, as is modern best practice for low-current mixed-signal systems \cite{WhatAreBasic}. Instead, the analogue and digital sections were physically partitioned, with sensitive analogue components and signal routed away from high-speed digital traces. Fencing vias were deployed along region boundaries to confine high-frequency digital currents and provide additional shielding for low-level analogue signals (Figure \ref{fig:via_fencing}).

Subsystems we're grouped together with jumpers connecting subsystems, allowing for easier debugging and testing. Care had to be taken in selecting the pin usage for both the STM32 and ESP32 as nearly all pins on both devices were used. 
% \todo{remove this}
% The small package size of the OPA3S328 (24-pin DSBGA with 0.4 mm pitch) also posed challenges for routing. With a minimum pad diameter of 0.25 mm and 0.4 mm pitch, the clearance between pads is only 0.15 mm, meaning that traces could not be routed between pads. Usually this would be solved using via-in-pads, however, the minimum via hole diameter of 0.3 mm meant that this was not possible with JLC's low cost PCB manufacturing process. By only using 2 of the 3 internal switches, OUTSB3 could be used to route the common node of switch B to the non inverting input of the buffer op-amp (Figure \ref{fig:bga_routing}). This reduced the number of gain settings to 2 as mentioned in section \ref{subsec:design_cur}. 

The complete PCB schematic is shown in Appendix \ref{appendix:schematic} and the final PCB layout in Figure \ref{fig:final_pcb}.

% \todo{Move to appendix}
% \begin{table}[H]
%     \centering
%     \caption{JLC PCB Standard Manufacturing Process Limits}
%     \label{tab:jlc_limits}
%     \begin{tabular}{ll}
%         \hline
%         \textbf{Parameter} & \textbf{Limit} \\
%         \hline
%         Minimum Trace Width & \SI{0.1}{\milli\meter} (4 mil) \\
%         Minimum Trace Spacing & \SI{0.1}{\milli\meter} (4 mil) \\
%         Minimum Via Diameter & \SI{0.45}{\milli\meter} \\
%         Minimum Via Hole Diameter & \SI{0.3}{\milli\meter} \\
%         Minimum BGA Pad Diameter & \SI{0.25}{\milli\meter} \\
%         Maximum Board Size & \SI{100}{\milli\meter} × \SI{100}{\milli\meter} \\
%         Number of Layers & 2, 4, or 6 layers \\
%         \hline
%     \end{tabular}
% \end{table}


% \todo{move to appendix}
% \begin{figure}[H]
%     \centering
%     \begin{minipage}{0.4\textwidth}
%         \centering
%         \includegraphics[width=\textwidth]{BioPal_render.png}
%         \caption{Final PCB Render}
%         \label{fig:final_pcb}
%     \end{minipage}\hfill
%     \begin{minipage}{0.35\textwidth}
%         \centering
%         \includegraphics[width=\textwidth]{BGA_Routing.png}
%         \caption{\\OPA3S328 BGA Routing Solution}
%         \label{fig:bga_routing}
%     \end{minipage}
% \end{figure}

\section{Firmware Development}
The embedded system architecture was driven by the \ac{POC} requirements of ease of use without technical expertise, versatility across diverse clinical settings and reliable measurements. To avoid dependence on external software and ensure the device can operate in any environment, all processing is performed on-device. 

Figure \ref{fig:system_overview} gives an overview of the system architecture. The STM32 serves as the measurement subsystem, interfacing with the analogue frontend to perform measurements. Results are then sent to the ESP32 via UART. The ESP32 acts as the system controller, managing calibration, sending commands to the STM32 and handling the user interfaces. An on-device LCD with buttons provides an interface for standalone operation in any setting, while an optional web-based interface accessible via \ac{BLE} enables enhanced usability when client devices are available.

\subsection{STM32 Measurement Subsystem}
\begin{figure}[H]
    \centering
    \includegraphics[width=\textwidth]{STMFlowDiagram.png}
    \caption{STM32 Measurement Subsystem Flowdiagram}
    \label{fig:stm32_flow}
\end{figure}
% Update flow diagram to show selection of MUX channel and gain settings
The STM32F303K8 was selected for its superior measurement capabilities compared to the ESP32. Its 12-bit \ac{DAC} and 12-bit \acp{ADC} provide 1 Msps sampling with hardware oversampling support, whilst advanced timers enable precise frequency generation with sub-microsecond accuracy. The ARM Cortex-M4F core includes a hardware \ac{FPU} and \ac{DSP} extensions, providing native support for \ac{FFT} operations through the CMSIS-DSP library. These capabilities enable the accurate impedance measurements required for biosensing applications.

The 38 measurement frequencies were carefully selected to ensure exact generation by the \ac{DAC}. Timer frequency generation on the STM32 follows Equation \ref{eq:timer_freq}, where $f_{TIM}$ is the timer clock frequency (64 MHz for TIM6), PSC is the prescaler register value, and ARR is the auto-reload register value. Both PSC and ARR are 16-bit registers with maximum values of 65535.

\begin{equation}
    f_{update} = \frac{f_{TIM}}{(PSC + 1) \times (ARR + 1)}
    \label{eq:timer_freq}
\end{equation}

For a target frequency $f_{target}$, suitable PSC and ARR values must be found such that $(PSC + 1) \times (ARR + 1) = f_{TIM} / f_{target}$ and $f_target = f_{meas} \times n_{oversampling}$. This constraint limits the achievable frequencies to those with integer divisors that fit within the 16-bit register limits. The selected frequencies follow roughly logarithmic spacing from 1 Hz to 100 kHz to provide adequate coverage of the biosensor's impedance spectrum.

TIM6 serves dual purposes, triggering both \ac{DAC} updates for signal generation and \ac{ADC} sampling for voltage and current measurements. This synchronous triggering is critical for accurate phase measurements, as any timing mismatch between current and voltage measurement would introduce phase errors. Oversampling improves \ac{DAC} resolution and reduces aliasing effects on the \ac{DUT}. At low frequencies, 128x oversampling is used (128 \ac{DAC} updates per sine wave cycle), limited by the maximum \ac{ADC} buffer size the STM32's 16 KB RAM can accommodate. At higher frequencies, oversampling was reduced due to the limited update speed of the peripherals. However, even at 100 kHz, 32x oversampling is maintained, well above the theoretical Nyquist limit of 2 samples per cycle.

\ac{DMA} channels enable simultaneous \ac{DAC} output and dual-channel \ac{ADC} sampling without CPU intervention. The STM32 controls the \ac{PGA} gain via SPI, switches the \ac{TIA} feedback resistor, and manages the relay multiplexer to select between the four \acp{DUT}.

Initially, a dynamic gain adjustment approach was implemented where the \ac{PGA} and \ac{TIA} gains were automatically adjusted based on measured signal amplitudes to maximise \ac{ADC} range utilisation without clipping. However, this resulted in large run-to-run variations and proved difficult to calibrate for. Fixed gain settings for each frequency were instead adopted based on the expected impedance response of the biosensor being measured, providing consistent and repeatable measurements.

The measurement process follows a state machine architecture. After receiving a start command from the ESP32 via UART, the STM32 sequences through each frequency for each \ac{DUT} in turn. For each measurement point, the system first generates the excitation signal and waits 1 second for the response to reach steady state. It then acquires 10 complete cycles of both voltage and current waveforms. These 10 cycles are then averaged to improve the signal-to-noise ratio before FFT processing. A 128-point FFT is performed on both the voltage and current signals using the ARM CMSIS-DSP library, extracting the magnitude and phase at the frequency of interest. After completing all 38 frequencies for a \ac{DUT}, the STM32 transmits the results to the ESP32 before proceeding to the next \ac{DUT}. This immediate transmission after each \ac{DUT} is necessary due to memory limitations.

\subsection{ESP32 User Interface and System Control}
\begin{figure}[H]
    \centering
    \includegraphics[width=\textwidth]{ESPFlowDiagram.png}
    \caption{ESP Flow Diagram}
    \label{fig:esp32_flow}
\end{figure}
The ESP32-C6 serves as the system controller, accepting user inputs, sending measurement commands to the STM32, calibrating the resulting measurements and displaying results to the user. The larger flash and RAM resources on the ESP32 allowed for the use of Free RTOS, using three concurrent tasks to handle different aspects of system operation. A UART reader task receives and parses data packets from the STM32 as they arrive. A data processor task performs impedance calculations and applies calibration corrections to the raw measurements. A GUI task manages the display updates, button inputs, and \ac{BLE} communication. This multi-tasking approach ensures responsive user interaction whilst processing measurement data in real-time.

The ESP32 orchestrates the complete measurement sequence. Upon receiving a start command from either the physical buttons or web interface, it configures the STM32 with the appropriate measurement parameters (number of \acp{DUT} and frequency range) and initiates the measurement cycle. As the STM32 send measurement data back via UART, the ESP32 applies calibration corrections and updates the display to show measurement progress. The calibration system corrects for systematic errors in the analogue frontend using reference measurements. After completing measurements for all \acp{DUT}, the ESP32 stores the results and makes them available for export via USB serial in a CSV format and sends the data over \ac{BLE} to the web interface if connected.

\subsubsection{On-Device Interface}
Physical buttons and a TFT LCD display enables complete standalone operation. Combined with battery power, this ensures the device functions in any resource-constrained setting without requiring external hardware. The high resolution display allows for an intuitive user interface. Users can configure the number of \acp{DUT} to measure, set the frequency range, initiate measurements, and view results without any external devices. This interface is critical for resource constrained \ac{POC} environments where reliable network connectivity or additional hardware might not be available.

\subsubsection{Web Interface via BLE}
A secondary web-based interface provides enhanced usability when client devices are available. A website was chosen over an app to ensure compatibility with all modern devices, regardless of operating system. This ensures that computers, phone or tablets can all be used to control the device. Three implementation approaches were considered for enabling remote control of the device via a web interface.

% Example of the web ui results page
\begin{figure}[H]
    \centering
    \includegraphics[width=0.8\textwidth]{BioPal WebUI.png}
    \caption{Web Interface Results Page}
    \label{fig:web_ui_results}
\end{figure}

\textbf{Option 1 - Existing WiFi Network:} The device connects to an existing network, enabling control over WiFi or the internet. While convenient once configured, this approach poses significant barriers in healthcare environments with strict network security policies due to the need to protect patient data. It also excludes settings without existing WiFi infrastructure, such as rural clinics.

\textbf{Option 2 - Device WiFi Hotspot:} The device hosts its own WiFi network, serving a local website. This avoids secure network requirements and functions everywhere, but forces client devices offline, potentially disrupting other applications and services.

\textbf{Option 3 - Web Application with BLE:} A web application hosted remotely connects to the device via \ac{BLE} usin the WebBLE protocol. This approach was selected as it provides the benefits of both previous options without their drawbacks. Client devices maintain internet connectivity, no secure network authentication is required, and the solution works universally. Critically, all measurement data transfers locally over \ac{BLE}, eliminating security risks and POPIA compliance concerns associated with transmitting patient information over the internet.

\section{System Integration}

\label{chap:design}