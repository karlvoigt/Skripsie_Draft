\graphicspath{{design/fig/},
                {introduction/fig/},
                {literature_review/fig/},
                {testing_and_validation/fig/},
                {appendices/fig/}}
\chapter{Project Planning Schedule}
\makeatletter\@mkboth{}{Appendix}\makeatother
\label{appen:derivations_bigramseg}

This is an appendix.

\chapter{Outcomes Compliance}
\makeatletter\@mkboth{}{Appendix}\makeatother
\label{appen:derivations_bigramseg2}

This is another appendix.

\chapter{Additional Hardware Design Information}
\makeatletter\@mkboth{}{Appendix}\makeatother
\label{appen:design}

\begin{figure}[H]
    \centering
    \includegraphics[width=0.5\textwidth]{INA_BW.jpeg}
    \caption[INA331 Bandwidth vs Gain]{INA331 Bandwidth vs Gain adapted from \cite{INA331}}
    \label{fig:ina_bw}
\end{figure}

\begin{figure}[H]
    \centering
    \includegraphics[width=0.5\textwidth]{KelvinSenseTIA.png}
    \caption[Kelvin Sense TIA]{Kelvin Sense TIA \cite{chioyeBuildProgrammableGain2021}}
    \label{fig:kelvin_sense_tia}
\end{figure}

%  design/fig/PhotoDiodeCircuit.png
%  equivalent circuit model of photodiode
\begin{figure}[H]
    \centering
    \includegraphics[width=0.5\textwidth]{PhotoDiodeCircuit.png}
    \caption[Equivalent Circuit Model of Photodiode]{Equivalent Circuit Model of Photodiode\cite{StabilizeYourTransimpedance}}
    \label{fig:photodiode}
\end{figure}

% photodiode tia equivalent circuit
\begin{figure}[H]
    \centering
    \includegraphics[width=0.5\textwidth]{PhotoTIA.png}
    \caption[Photodiode TIA Equivalent Circuit]{Photodiode TIA Equivalent Circuit}
    \label{fig:tia_photo}
\end{figure}

\begin{figure}[H]
    \centering
    \begin{subfigure}[b]{\textwidth}
        \centering
        \includegraphics[width=\textwidth]{MatLabBeta_37,5.png}
        \caption{$R_F=37.5~\Omega$}
        \label{fig:matlab_beta_37.5}
    \end{subfigure}
    
    \vspace{1em}
    
    \begin{subfigure}[b]{\textwidth}
        \centering
        \includegraphics[width=\textwidth]{MatLabBeta_7500.png}
        \caption{$R_F=7.5~k\Omega$}
        \label{fig:matlab_beta_7.5}
    \end{subfigure}
    \caption{$\frac{1}{\beta(j\omega)}$ Frequency Responses}
    \label{fig:matlab_beta}
\end{figure}
\chapter{Additional Software Design Information}
\makeatletter\@mkboth{}{Appendix}\makeatother
\label{appen:software}

\begingroup
\small % adjust size to fit; change to \scriptsize if needed
\setlength{\LTpre}{0pt}
\setlength{\LTpost}{6pt}
\begin{longtable}{lrrrr}
    \caption{Configured Sampling Frequencies and Timer Settings for 64 MHz Peripheral Clock}
    \label{tab:frequency_table}\\
    \toprule
    \textbf{Signal Freq (Hz)} & \textbf{PSC} & \textbf{ARR} & \textbf{Oversampling} & \textbf{Sampling Freq (Hz)}\\
    \midrule
    \endfirsthead

    \caption[]{Configured Sampling Frequencies and Timer Settings for 64 MHz Peripheral Clock (continued)}\\
    \toprule
    \textbf{Signal Freq (Hz)} & \textbf{PSC} & \textbf{ARR} & \textbf{Oversampling} & \textbf{Sampling Freq (Hz)}\\
    \midrule
    \endhead

    \midrule
    \multicolumn{5}{r}{\emph{Continued on next page}}\\
    \endfoot

    \bottomrule
    \endlastfoot

    100000 & 0 & 19    & 32  & 3,200,000 \\
     80000 & 0 & 24    & 32  & 2,560,000 \\
     62500 & 0 & 31    & 32  & 2,000,000 \\
     50000 & 0 & 19    & 64  & 3,200,000 \\
     25000 & 0 & 19    & 128 & 3,200,000 \\
     15625 & 0 & 31    & 128 & 2,000,000 \\
     12500 & 0 & 39    & 128 & 1,600,000 \\
     10000 & 0 & 49    & 128 & 1,280,000 \\
      6250 & 0 & 79    & 128 &   800,000 \\
      5000 & 0 & 99    & 128 &   640,000 \\
      4000 & 0 & 124   & 128 &   512,000 \\
      3125 & 0 & 159   & 128 &   400,000 \\
      2500 & 0 & 199   & 128 &   320,000 \\
      2000 & 0 & 249   & 128 &   256,000 \\
      1250 & 0 & 399   & 128 &   160,000 \\
      1000 & 0 & 499   & 128 &   128,000 \\
       800 & 0 & 624   & 128 &   102,400 \\
       625 & 0 & 799   & 128 &    80,000 \\
       500 & 0 & 999   & 128 &    64,000 \\
       400 & 0 & 1249  & 128 &    51,200 \\
       250 & 0 & 1999  & 128 &    32,000 \\
       200 & 0 & 2499  & 128 &    25,600 \\
       160 & 0 & 3124  & 128 &    20,480 \\
       125 & 0 & 3999  & 128 &    16,000 \\
       100 & 0 & 4999  & 128 &    12,800 \\
        80 & 0 & 6249  & 128 &    10,240 \\
        50 & 0 & 9999  & 128 &     6,400 \\
        40 & 0 & 12499 & 128 &     5,120 \\
        32 & 0 & 15624 & 128 &     4,096 \\
        25 & 0 & 19999 & 128 &     3,200 \\
        20 & 0 & 24999 & 128 &     2,560 \\
        16 & 0 & 31249 & 128 &     2,048 \\
        10 & 0 & 49999 & 128 &     1,280 \\
         8 & 0 & 62499 & 128 &     1,024 \\
         5 & 1 & 49999 & 128 &       640 \\
         4 & 1 & 62499 & 128 &       512 \\
         2 & 3 & 62499 & 128 &       256 \\
         1 & 7 & 62499 & 128 &       128 \\
\end{longtable}
\endgroup

\begingroup
\small
\setlength{\LTpre}{0pt}
\setlength{\LTpost}{6pt}
\begin{longtable}{l l r r}
    \caption{TIA and PGA Gain Settings per Signal Frequency. TIA: 0=High, 1=Low. PGA index maps to gain \{0=1,1=2,2=5,3=10,4=20,5=50,6=100,7=200\}.}
    \label{tab:gain_settings_table}\\
    \toprule
    \textbf{Signal Freq (Hz)} & \textbf{TIA} & \textbf{PGA idx} & \textbf{PGA gain} \\
    \midrule
    \endfirsthead

    \caption[]{TIA and PGA Gain Settings per Signal Frequency (continued)}\\
    \toprule
    \textbf{Signal Freq (Hz)} & \textbf{TIA} & \textbf{PGA idx} & \textbf{PGA gain} \\
    \midrule
    \endhead

    \midrule
    \multicolumn{4}{r}{\emph{Continued on next page}}\\
    \endfoot

    \bottomrule
    \endlastfoot

    100000 & Low  & 1 & 2   \\
     80000 & Low  & 1 & 2   \\
     62500 & Low  & 1 & 2   \\
     50000 & Low  & 1 & 2   \\
     25000 & Low  & 2 & 5   \\
     15625 & Low  & 2 & 5   \\
     12500 & Low  & 2 & 5   \\
     10000 & Low  & 2 & 5   \\
      6250 & Low  & 2 & 5   \\
      5000 & Low  & 3 & 10  \\
      4000 & Low  & 3 & 10  \\
      3125 & Low  & 3 & 10  \\
      2500 & Low  & 3 & 10  \\
      2000 & Low  & 4 & 20  \\
      1250 & Low  & 4 & 20  \\
      1000 & Low  & 4 & 20  \\
       800 & Low  & 5 & 50  \\
       625 & Low  & 5 & 50  \\
       500 & Low  & 5 & 50  \\
       400 & Low  & 6 & 100 \\
       250 & Low  & 6 & 100 \\
       200 & Low  & 6 & 100 \\
       160 & High & 0 & 1   \\
       125 & High & 0 & 1   \\
       100 & High & 1 & 2   \\
        80 & High & 1 & 2   \\
        50 & High & 1 & 2   \\
        40 & High & 1 & 2   \\
        32 & High & 1 & 2   \\
        25 & High & 2 & 5   \\
        20 & High & 2 & 5   \\
        16 & High & 2 & 5   \\
        10 & High & 2 & 5   \\
         8 & High & 2 & 5   \\
         5 & High & 3 & 10  \\
         4 & High & 3 & 10  \\
         2 & High & 3 & 10  \\
         1 & High & 3 & 10  \\
\end{longtable}
\endgroup

\chapter{Bill of Materials}

\begin{table}[H]
    \centering
    \caption{Bill of Materials and Cost Breakdown}
    \label{tab:bom_cost}
    \begin{tabular}{llrr}
        \hline
        \textbf{Description} & \textbf{Part No.} & \textbf{Price per Unit} & \textbf{Units} \\
        \hline
        STM32F303 & NUCLEO-F303K8 & R195.96 & 1 \\
        FireBeetle 2 ESP32-C6 & DFR1075 & R144.90 & 1 \\
        2.4" Display & W18366 & R243.80 & 1 \\
        Battery & 2000mAh LiPo & R110.40 & 1 \\
        Keypad & Adafruit 5001 & R158.86 & 1 \\
        General Opamp & TLV9061IDBVR & R7.35 & 4 \\
        Anti-Aliasing Filter & LTC1069 & R256.31 & 1 \\
        Instrumentation Amplifier & INA331 & R38.52 & 1 \\
        TIA & OPA3S328 & R162.06 & 1 \\
        PGA & PGA113 & R51.48 & 1 \\
        5V Regulator & TPS61071 & R20.77 & 1 \\
        Latching Signal Relay & TXS2-L2-3V & R81.47 & 3 \\
        Relay Driver BJT & BC847BS & R3.55 & 3 \\
        1k Resistor Array & ACASN1001S1001P1AT & R12.78 & 1 \\
        DUT Connectors & 2 POS 1.27MM PITCH & R19.38 & 4 \\
        Shunt Jumpers & 2 way 2.54mm & R0.42 & 25 \\
        Enclosure & Gianta ABS Enclosure & R201.25 & 1 \\
        PCB & JLC PCB & R14.00 & 1 \\
        Other Passive Components & N/A & R170.12 & 1 \\
        \hline
        \multicolumn{3}{r}{\textbf{Total Cost:}} & \textbf{R1,983.55} \\
        \hline
    \end{tabular}
\end{table}
\begin{flushleft}
    \small
    Note: Shipping costs excluded as they are not directly proportional to units produced.
\end{flushleft}

% Passive component values

\chapter{Additional Simulation Results}
\makeatletter\@mkboth{}{Appendix}\makeatother
\label{appen:simulation}

\begin{figure}[H]
    \centering
    \includegraphics[width=\textwidth]{CPE_linear.png}
    \caption{Simulated CPE Frequency Response}
    \label{fig:cpe_sim}
\end{figure}
\begin{figure}[H]
        \centering
        \includegraphics[width=\textwidth]{DUT_Z.png}
        \caption{Simulated IDE Frequency Response}
        \label{fig:ide_freq}
\end{figure}

\begin{figure}[H]
    \centering
    \begin{subfigure}[b]{0.48\textwidth}
        \centering
        \includegraphics[width=\textwidth]{FIR_10k.png}
        \caption{10kHz Cutoff Frequency}
        \label{fig:fir_10k}
    \end{subfigure}\hfill
    \begin{subfigure}[b]{0.48\textwidth}
        \centering
        \includegraphics[width=\textwidth]{FIR_100k.png}
        \caption{100kHz Cutoff Frequency}
        \label{fig:fir_100k}
    \end{subfigure}
    \caption{Simulated FIR filter Frequency Responses}
    \label{fig:lt_aa_freq}
\end{figure}
\begin{figure}[H]
    \centering
    \begin{subfigure}[b]{0.4\textwidth}
        \centering
        \includegraphics[width=\textwidth]{LTC_10k.png}
        \caption{10kHz Cutoff Frequency}
        \label{fig:ltc_10k}
    \end{subfigure}\hfill
    \begin{subfigure}[b]{0.4\textwidth}
        \centering
        \includegraphics[width=\textwidth]{LTC_100k.png}
        \caption{100kHz Cutoff Frequency}
        \label{fig:ltc_100k}
    \end{subfigure}
    \caption{LTC1069-7 Datasheet Frequency Responses}
    \label{fig:ltc_aa_freq}
\end{figure}
% design/fig/ExcitationFilteringFFT.png Showing the effect of filtering on DAC output in reducing aliasing
\begin{figure}[H]
    \centering
    \includegraphics[width=0.8\textwidth]{ExcitationFilteringFFT.png}
    \caption{FFT of Simulated Excitation Signal Filtering showing reduction in aliasing artefacts}
    \label{fig:excitation_filtering}
\end{figure}
% Two horizontal figures, not subfigures but minipages with the  ina33 and tlv gain stages freq responses design/fig/TLVfreqResponse.png design/fig/INA331_FreqResponse.png
\begin{figure}[H]
    \centering
    \begin{minipage}{0.48\textwidth}
        \centering
        \includegraphics[width=\textwidth]{INA331_FreqResponse.png}
        \caption{Simulated INA331 Frequency Response}
        \label{fig:ina_freq}
    \end{minipage}\hfill
    \begin{minipage}{0.48\textwidth}
        \centering
        \includegraphics[width=\textwidth]{TLVfreqResponse.png}
        \caption{Simulated TLV9061 Frequency Response}
        \label{fig:tlv_freq}
    \end{minipage}
    \caption{Simulated Gain Stage Frequency Responses}
    \label{fig:gain_stage_freq}
\end{figure}
% Insert design/fig/TIAFreqResponse_7500.png design/fig/TIAFreqResponse_37,5.png which shows the frequency response of the TIA at both gain settings as horizontally stacked subfigures
\begin{figure}[H]
    \centering
    \begin{subfigure}{0.48\textwidth}
        \centering
        \includegraphics[width=\textwidth]{TIAFreqResponse_37,5.png}
        \caption{$R_F = 37.5 \Omega$}
        \label{fig:tia_freq_37.5}
    \end{subfigure}
    \hfill
    \begin{subfigure}{0.48\textwidth}
        \centering
        \includegraphics[width=\textwidth]{TIAFreqResponse_7500.png}
        \caption{$R_F = 7.5~k\Omega$}
        \label{fig:tia_freq_7.5k}
    \end{subfigure}
    \caption{Simulated TIA Frequency Response}
    \label{fig:tia_freq}
\end{figure}
% Insert time domain response of the full current measurement stage design/fig/FullCurrentMeasurementStage.png
\begin{figure}[H]
    \centering
    \includegraphics[width=0.8\textwidth]{FullCurrentMeasurementStage.png}
    \caption{Simulated Time Domain Response of Full Current Measurement Stage}
    \label{fig:full_current_measurement}
\end{figure}
% Insert two subfigures in the figure that compares current measurement with and without AA filter design/fig/SampledTIA_AAvsNoAA.png design/fig/IDECurrentAAvsNoAA.png
\begin{figure}[H]
    \centering
    \begin{subfigure}[b]{0.8\textwidth}
        \centering
        \includegraphics[width=\textwidth]{IDECurrentAAvsNoAA.png}
        \caption{Current Through IDE}
        \label{fig:ide_current_aa_vs_noaa}
    \end{subfigure}
    
    \vspace{1em}
    
    \begin{subfigure}[b]{0.8\textwidth}
        \centering
        \includegraphics[width=\textwidth]{SampledTIA_AAvsNoAA.png}
        \caption{Sampled TIA Output}
        \label{fig:tia_aa_vs_noaa}
    \end{subfigure}
    \caption{Simulated Current Measurement with and without Anti-Aliasing Filter}
    \label{fig:current_measurement_aa_vs_noaa}
\end{figure}


% \begin{figure}[H]
%     \centering
%     \begin{subfigure}[b]{0.6\textwidth}
%         \centering
%         \includegraphics[width=\textwidth]{TIA_Freq_37,5.png}
%         \caption{$R_F = 37.5 \Omega$}
%         \label{fig:tia_freq_37.5}
%     \end{subfigure}
    
%     \vspace{1em}
    
%     \begin{subfigure}[b]{0.6\textwidth}
%         \centering
%         \includegraphics[width=\textwidth]{TIA_Freq_7500.png}
%         \caption{$R_F = 7.5~k\Omega$}
%         \label{fig:tia_freq_7.5k}
%     \end{subfigure}
%     \caption{Simulated TIA Frequency Response}
%     \label{fig:tia_freq}
% \end{figure}

\chapter{Full PCB Layouts and Additional PCB Information}
\makeatletter\@mkboth{}{Appendix}\makeatother
\label{appen:pcb}

\begin{table}[H]
    \centering
    \caption{JLC PCB Standard Manufacturing Process Limits}
    \label{tab:jlc_limits}
    \begin{tabular}{ll}
        \hline
        \textbf{Parameter} & \textbf{Limit} \\
        \hline
        Minimum Trace Width & \SI{0.1}{\milli\meter} (4 mil) \\
        Minimum Trace Spacing & \SI{0.1}{\milli\meter} (4 mil) \\
        Minimum Via Diameter & \SI{0.45}{\milli\meter} \\
        Minimum Via Hole Diameter & \SI{0.3}{\milli\meter} \\
        Minimum BGA Pad Diameter & \SI{0.25}{\milli\meter} \\
        Maximum Board Size & \SI{100}{\milli\meter} × \SI{100}{\milli\meter} \\
        Number of Layers & 2, 4, or 6 layers \\
        \hline
    \end{tabular}
\end{table}

\begin{figure}[H]
    \centering
    \begin{minipage}{0.5\textwidth}
        \centering
        \includegraphics[width=\textwidth]{BioPal_render.png}
        \caption{Final PCB Render}
        \label{fig:final_pcb}
    \end{minipage}\hfill
    \begin{minipage}{0.5\textwidth}
        \centering
        \includegraphics[width=\textwidth]{BGA_Routing.png}
        \caption{\\OPA3S328 BGA Routing Solution}
        \label{fig:bga_routing}
    \end{minipage}
\end{figure}

\begin{figure}[H]
    \centering
    \begin{subfigure}[b]{0.75\textwidth}
        \centering
        \includegraphics[width=\textwidth]{BioPal_Front.png}
        \caption{PCB front layout}
        \label{fig:pcb_front}
    \end{subfigure}
    
    \vspace{1em}
    
    \begin{subfigure}[b]{0.75\textwidth}
        \centering
        \includegraphics[width=\textwidth]{BioPal_Back.png}
        \caption{PCB back layout}
        \label{fig:pcb_back}
    \end{subfigure}
    \caption{Final PCB Layout}
    \label{fig:pcb_layout}
\end{figure}

\chapter{Measured Frequency Responses and Calibration Data}
\makeatletter\@mkboth{}{Appendix}\makeatother
\label{appen:testing}

% Add measured frequency responses of each gain stage (with LOESS filtering applied in matlab)
\begin{figure}[H]
    \centering
    \begin{minipage}[t]{0.48\textwidth}
        \vspace{0pt}
        \centering
        \includegraphics[width=\textwidth]{SmoothedBodeV.png}
        \caption{INA331 Measured Frequency Response with Filtered Calibration Curve}
        \label{fig:smoothed_bode_v}
    \end{minipage}\hfill
    \begin{minipage}[t]{0.48\textwidth}
        \vspace{0pt}
        \centering
        \includegraphics[width=\textwidth]{SmoothedBodeTIA37,5.png}
        \caption{TIA Measured Frequency Response at $R_F = 37.5~\Omega$ with Filtered Calibration Curve}
        \label{fig:smoothed_bode_i_37.5}
    \end{minipage}
\end{figure}

\begin{figure}[H]
    \centering
    \begin{minipage}[t]{0.48\textwidth}
        \vspace{0pt}
        \centering
        \includegraphics[width=\textwidth]{SmoothedBodeTIA7500.png}
        \caption{TIA Measured Frequency Response at $R_F = 7.5~k\Omega$ with Filtered Calibration Curve}
        \label{fig:smoothed_bode_i_7.5k}
    \end{minipage}\hfill
    \begin{minipage}[t]{0.48\textwidth}
        \vspace{0pt}
        \centering
        \includegraphics[width=\textwidth]{SmoothedBodePGA200.png}
        \caption{PGA113 Measured Frequency Response at Gain 200 with Filtered Calibration Curve}
        \label{fig:smoothed_bode_pga_200}
    \end{minipage}
\end{figure}

\begin{figure}[H]
    \centering
    \begin{minipage}[t]{0.48\textwidth}
        \vspace{0pt}
        \centering
        \includegraphics[width=\textwidth]{SmoothedBodePGA100.png}
        \caption{PGA113 Measured Frequency Response at Gain 100 with Filtered Calibration Curve}
        \label{fig:smoothed_bode_pga_100}
    \end{minipage}\hfill
    \begin{minipage}[t]{0.48\textwidth}
        \vspace{0pt}
        \centering
        \includegraphics[width=\textwidth]{SmoothedBodePGA50.png}
        \caption{PGA113 Measured Frequency Response at Gain 50 with Filtered Calibration Curve}
        \label{fig:smoothed_bode_pga_50}
    \end{minipage}
\end{figure}

\begin{figure}[H]
    \centering
    \begin{minipage}[t]{0.48\textwidth}
        \vspace{0pt}
        \centering
        \includegraphics[width=\textwidth]{SmoothedBodePGA20.png}
        \caption{PGA113 Measured Frequency Response at Gain 20 with Filtered Calibration Curve}
        \label{fig:smoothed_bode_pga_20}
    \end{minipage}\hfill
    \begin{minipage}[t]{0.48\textwidth}
        \vspace{0pt}
        \centering
        \includegraphics[width=\textwidth]{SmoothedBodePGA10.png}
        \caption{PGA113 Measured Frequency Response at Gain 10 with Filtered Calibration Curve}
        \label{fig:smoothed_bode_pga_10}
    \end{minipage}
\end{figure}

\begin{figure}[H]
    \centering
    \begin{minipage}[t]{0.48\textwidth}
        \vspace{0pt}
        \centering
        \includegraphics[width=\textwidth]{SmoothedBodePGA5.png}
        \caption{PGA113 Measured Frequency Response at Gain 5 with Filtered Calibration Curve}
        \label{fig:smoothed_bode_pga_5}
    \end{minipage}\hfill
    \begin{minipage}[t]{0.48\textwidth}
        \vspace{0pt}
        \centering
        \includegraphics[width=\textwidth]{SmoothedBodePGA2.png}
        \caption{PGA113 Measured Frequency Response at Gain 2 with Filtered Calibration Curve}
        \label{fig:smoothed_bode_pga_2}
    \end{minipage}
\end{figure}

\begin{figure}[H]
    \centering
    \begin{minipage}[t]{0.48\textwidth}
        \vspace{0pt}
        \centering
        \includegraphics[width=\textwidth]{SmoothedBodePGA1.png}
        \caption{PGA113 Measured Frequency Response at Gain 1 with Filtered Calibration Curve}
        \label{fig:smoothed_bode_pga_1}
    \end{minipage}\hfill
    \begin{minipage}[t]{0.48\textwidth}
        % intentionally left empty to keep consistent two-column layout for odd number of figures
    \end{minipage}
\end{figure}

\chapter{Run to Run Variation Data}
\makeatletter\@mkboth{}{Appendix}\makeatother
\label{appen:variance}

% Run to run variance table
\begin{table}[H]
    \centering
    \caption{Run to Run Variation Statistics}
    \label{tab:run_to_run_stats}
    \begin{tabular}{rrrr}
        \hline
        \textbf{Frequency} & \textbf{Magnitude Standard} & \textbf{Normalised Standard } & \textbf{Phase Standard} \\
        \textbf{(Hz)} & \textbf{Deviation ($\Omega$)} & \textbf{Deviation (\%)} & \textbf{Deviation (°)} \\
        \hline
        100000 & 0.031 & 0.32\% & 0.183 \\
        80000 & 0.027 & 0.30\% & 0.219 \\
        62500 & 0.009 & 0.10\% & 0.041 \\
        50000 & 0.020 & 0.21\% & 0.113 \\
        25000 & 0.012 & 0.11\% & 0.102 \\
        15625 & 0.007 & 0.06\% & 0.052 \\
        12500 & 0.009 & 0.07\% & 0.036 \\
        10000 & 0.012 & 0.08\% & 0.030 \\
        6250 & 0.022 & 0.10\% & 0.030 \\
        5000 & 0.026 & 0.10\% & 0.028 \\
        4000 & 0.031 & 0.10\% & 0.048 \\
        3125 & 0.050 & 0.12\% & 0.040 \\
        2500 & 0.069 & 0.14\% & 0.027 \\
        2000 & 0.121 & 0.20\% & 0.110 \\
        1250 & 0.182 & 0.18\% & 0.174 \\
        1000 & 0.221 & 0.17\% & 0.096 \\
        800 & 0.241 & 0.14\% & 0.059 \\
        625 & 1.930 & 0.83\% & 0.651 \\
        500 & 1.561 & 0.49\% & 0.408 \\
        400 & 2.297 & 0.52\% & 0.210 \\
        250 & 14.929 & 1.39\% & 0.848 \\
        200 & 48.305 & 2.63\% & 1.667 \\
        160 & 300.174 & 5.62\% & 2.962 \\
        125 & 1597.204 & 10.79\% & 6.976 \\
        100 & 4606.785 & 25.22\% & 14.699 \\
        80 & 13895.325 & 45.70\% & 56.690 \\
        50 & 32077.954 & 62.80\% & 93.550 \\
        40 & 17920.418 & 45.73\% & 100.532 \\
        32 & 27779.397 & 54.32\% & 94.192 \\
        \hline
    \end{tabular}
\end{table}

% Three subfigures with Magnitude STndard deviation plotted from 100kHz to 32,125 and 160 Hz respectiv ely
\begin{figure}[H]
    \centering
    \begin{subfigure}[b]{0.8\textwidth}
        \centering
        \includegraphics[width=\textwidth]{StdDev32-100k.png}
        \caption{32Hz to 100kHz}
        \label{fig:run_to_run_mag_std_100k_32}
    \end{subfigure}
    
    \vspace{1em}
    
    \begin{subfigure}[b]{0.8\textwidth}
        \centering
        \includegraphics[width=\textwidth]{StdDev125-100k.png}
        \caption{125Hz to 100kHz}
        \label{fig:run_to_run_mag_std_100k_125}
    \end{subfigure}
    
    \vspace{1em}
    
    \begin{subfigure}[b]{0.8\textwidth}
        \centering
        \includegraphics[width=\textwidth]{StdDev160-100k.png}
        \caption{160Hz to 100kHz}
        \label{fig:run_to_run_mag_std_100k_160}
    \end{subfigure}
    \caption{Run to Run Magnitude Standard Deviation at Various Frequency Ranges}
    \label{fig:run_to_run_mag_std}
\end{figure}