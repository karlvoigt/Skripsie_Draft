\graphicspath{{design/fig/},
                {introduction/fig/},
                {literature_review/fig/},
                {testing_and_validation/fig/},
                {appendices/fig/}}
\chapter{Project Planning Schedule}
\makeatletter\@mkboth{}{Appendix}\makeatother
\label{appen:derivations_bigramseg}

This is an appendix.

\chapter{Outcomes Compliance}
\makeatletter\@mkboth{}{Appendix}\makeatother
\label{appen:derivations_bigramseg2}

This is another appendix.

\chapter{Additional Design Information}
\makeatletter\@mkboth{}{Appendix}\makeatother
\label{appen:design}

\begin{figure}[H]
    \centering
    \includegraphics[width=0.5\textwidth]{INA_BW.jpeg}
    \caption[INA331 Bandwidth vs Gain]{INA331 Bandwidth vs Gain adapted from \cite{INA331}}
    \label{fig:ina_bw}
\end{figure}

\begin{figure}[]
    \centering
    \includegraphics[width=0.8\textwidth]{KelvinSenseTIA.png}
    \caption[Example of TIA Kelvin Sense Switch Connections and High Impedance Buffer]{Example of TIA Kelvin Sense Switch Connections and High Impedance Buffer\cite{chioyeBuildProgrammableGain2021}}
    \label{fig:kelvin_sense_tia}
\end{figure}

%  design/fig/PhotoDiodeCircuit.png
%  equivalent circuit model of photodiode
\begin{figure}[H]
    \centering
    \includegraphics[width=0.5\textwidth]{PhotoDiodeCircuit.png}
    \caption[Equivalent Circuit Model of Photodiode]{Equivalent Circuit Model of Photodiode\cite{StabilizeYourTransimpedance}}
    \label{fig:photodiode}
\end{figure}

% photodiode tia equivalent circuit
\begin{figure}[H]
    \centering
    \includegraphics[width=0.5\textwidth]{PhotoTIA.png}
    \caption[Photodiode TIA Equivalent Circuit]{Photodiode TIA Equivalent Circuit}
    \label{fig:tia_photo}
\end{figure}

\begin{figure}[H]
    \centering
    \begin{subfigure}[b]{\textwidth}
        \centering
        \includegraphics[width=\textwidth]{MatLabBeta_37,5.png}
        \caption{$R_F=37.5~\Omega$}
        \label{fig:matlab_beta_37.5}
    \end{subfigure}
    
    \vspace{1em}
    
    \begin{subfigure}[b]{\textwidth}
        \centering
        \includegraphics[width=\textwidth]{MatLabBeta_7500.png}
        \caption{$R_F=7.5~k\Omega$}
        \label{fig:matlab_beta_7.5}
    \end{subfigure}
    \caption{$\frac{1}{\beta(j\omega)}$ Frequency Responses}
    \label{fig:matlab_beta}
\end{figure}

\begin{table}[H]
    \centering
    \caption{Bill of Materials and Cost Breakdown}
    \label{tab:bom_cost}
    \begin{tabular}{llrr}
        \hline
        \textbf{Description} & \textbf{Part No.} & \textbf{Price per Unit} & \textbf{Units} \\
        \hline
        STM32F303 & NUCLEO-F303K8 & R195.96 & 1 \\
        FireBeetle 2 ESP32-C6 & DFR1075 & R144.90 & 1 \\
        2.4" Display & W18366 & R243.80 & 1 \\
        Battery & 2000mAh LiPo & R110.40 & 1 \\
        Keypad & Adafruit 5001 & R158.86 & 1 \\
        General Opamp & TLV9061IDBVR & R7.35 & 4 \\
        Anti-Aliasing Filter & LTC1069 & R256.31 & 1 \\
        Instrumentation Amplifier & INA331 & R38.52 & 1 \\
        TIA & OPA3S328 & R162.06 & 1 \\
        PGA & PGA113 & R51.48 & 1 \\
        5V Regulator & TPS61071 & R20.77 & 1 \\
        Latching Signal Relay & TXS2-L2-3V & R81.47 & 3 \\
        Relay Driver BJT & BC847BS & R3.55 & 3 \\
        1k Resistor Array & ACASN1001S1001P1AT & R12.78 & 1 \\
        DUT Connectors & 2 POS 1.27MM PITCH & R19.38 & 4 \\
        Shunt Jumpers & 2 way 2.54mm & R0.42 & 25 \\
        Enclosure & Gianta ABS Enclosure & R201.25 & 1 \\
        PCB & JLC PCB & R14.00 & 1 \\
        Other Passive Components & N/A & R170.12 & 1 \\
        \hline
        \multicolumn{3}{r}{\textbf{Total Cost:}} & \textbf{R1,983.55} \\
        \hline
    \end{tabular}
\end{table}

\chapter{Additional Simulation Results}
\makeatletter\@mkboth{}{Appendix}\makeatother
\label{appen:simulation}

\begin{figure}[H]
        \centering
        \includegraphics[width=\textwidth]{CPE_linear.png}
        \caption{Simulated CPE Frequency Response}
        \label{fig:cpe_sim}
\end{figure}
\begin{figure}[H]
        \centering
        \includegraphics[width=\textwidth]{DUT_Z.png}
        \caption{Simulated IDE Frequency Response}
        \label{fig:ide_freq}
\end{figure}

\begin{figure}[H]
    \centering
    \begin{subfigure}[b]{0.48\textwidth}
        \centering
        \includegraphics[width=\textwidth]{FIR_10k.png}
        \caption{10kHz Cutoff Frequency}
        \label{fig:fir_10k}
    \end{subfigure}\hfill
    \begin{subfigure}[b]{0.48\textwidth}
        \centering
        \includegraphics[width=\textwidth]{FIR_100k.png}
        \caption{100kHz Cutoff Frequency}
        \label{fig:fir_100k}
    \end{subfigure}
    \caption{Simulated FIR filter Frequency Responses}
    \label{fig:lt_aa_freq}
\end{figure}

\begin{figure}[H]
    \centering
    \begin{subfigure}[b]{0.4\textwidth}
        \centering
        \includegraphics[width=\textwidth]{LTC_10k.png}
        \caption{10kHz Cutoff Frequency}
        \label{fig:ltc_10k}
    \end{subfigure}\hfill
    \begin{subfigure}[b]{0.4\textwidth}
        \centering
        \includegraphics[width=\textwidth]{LTC_100k.png}
        \caption{100kHz Cutoff Frequency}
        \label{fig:ltc_100k}
    \end{subfigure}
    \caption{LTC1069-7 Datasheet Frequency Responses}
    \label{fig:ltc_aa_freq}
\end{figure}

% design/fig/ExcitationFilteringFFT.png Showing the effect of filtering on DAC output in reducing aliasing
\begin{figure}[H]
    \centering
    \includegraphics[width=0.8\textwidth]{ExcitationFilteringFFT.png}
    \caption{FFT of Simulated Excitation Signal Filtering showing reduction in aliasing artefacts}
    \label{fig:excitation_filtering}
\end{figure}

% Two horizontal figures, not subfigures but minipages with the  ina33 and tlv gain stages freq responses design/fig/TLVfreqResponse.png design/fig/INA331_FreqResponse.png
\begin{figure}[H]
    \centering
    \begin{minipage}{0.48\textwidth}
        \centering
        \includegraphics[width=\textwidth]{INA331_FreqResponse.png}
        \caption{Simulated INA331 Frequency Response}
        \label{fig:ina_freq}
    \end{minipage}\hfill
    \begin{minipage}{0.48\textwidth}
        \centering
        \includegraphics[width=\textwidth]{TLVfreqResponse.png}
        \caption{Simulated TLV9061 Frequency Response}
        \label{fig:tlv_freq}
    \end{minipage}
    \caption{Simulated Gain Stage Frequency Responses}
    \label{fig:gain_stage_freq}
\end{figure}

% Insert two subfigures in the figure that compares current measurement with and without AA filter design/fig/SampledTIA_AAvsNoAA.png design/fig/IDECurrentAAvsNoAA.png
\begin{figure}[H]
    \centering
    \begin{subfigure}[b]{0.8\textwidth}
        \centering
        \includegraphics[width=\textwidth]{IDECurrentAAvsNoAA.png}
        \caption{Current Through IDE}
        \label{fig:ide_current_aa_vs_noaa}
    \end{subfigure}
    
    \vspace{1em}
    
    \begin{subfigure}[b]{0.8\textwidth}
        \centering
        \includegraphics[width=\textwidth]{SampledTIA_AAvsNoAA.png}
        \caption{Sampled TIA Output}
        \label{fig:tia_aa_vs_noaa}
    \end{subfigure}
    \caption{Simulated Current Measurement with and without Anti-Aliasing Filter}
    \label{fig:current_measurement_aa_vs_noaa}
\end{figure}

% Insert design/fig/TIAFreqResponse_7500.png design/fig/TIAFreqResponse_37,5.png which shows the frequency response of the TIA at both gain settings as horizontally stacked subfigures
\begin{figure}[H]
    \centering
    \begin{subfigure}{0.48\textwidth}
        \centering
        \includegraphics[width=\textwidth]{TIAFreqResponse_37,5.png}
        \caption{$R_F = 37.5 \Omega$}
        \label{fig:tia_freq_37.5}
    \end{subfigure}
    \hfill
    \begin{subfigure}{0.48\textwidth}
        \centering
        \includegraphics[width=\textwidth]{TIAFreqResponse_7500.png}
        \caption{$R_F = 7.5~k\Omega$}
        \label{fig:tia_freq_7.5k}
    \end{subfigure}
    \caption{Simulated TIA Frequency Response}
    \label{fig:tia_freq}
\end{figure}


% Insert time domain response of the full current measurement stage design/fig/FullCurrentMeasurementStage.png
\begin{figure}[H]
    \centering
    \includegraphics[width=0.8\textwidth]{FullCurrentMeasurementStage.png}
    \caption{Simulated Time Domain Response of Full Current Measurement Stage}
    \label{fig:full_current_measurement}
\end{figure}

% \begin{figure}[H]
%     \centering
%     \begin{subfigure}[b]{0.6\textwidth}
%         \centering
%         \includegraphics[width=\textwidth]{TIA_Freq_37,5.png}
%         \caption{$R_F = 37.5 \Omega$}
%         \label{fig:tia_freq_37.5}
%     \end{subfigure}
    
%     \vspace{1em}
    
%     \begin{subfigure}[b]{0.6\textwidth}
%         \centering
%         \includegraphics[width=\textwidth]{TIA_Freq_7500.png}
%         \caption{$R_F = 7.5~k\Omega$}
%         \label{fig:tia_freq_7.5k}
%     \end{subfigure}
%     \caption{Simulated TIA Frequency Response}
%     \label{fig:tia_freq}
% \end{figure}

\chapter{Full PCB Layouts and Additional PCB Information}
\makeatletter\@mkboth{}{Appendix}\makeatother
\label{appen:pcb}

\begin{table}[H]
    \centering
    \caption{JLC PCB Standard Manufacturing Process Limits}
    \label{tab:jlc_limits}
    \begin{tabular}{ll}
        \hline
        \textbf{Parameter} & \textbf{Limit} \\
        \hline
        Minimum Trace Width & \SI{0.1}{\milli\meter} (4 mil) \\
        Minimum Trace Spacing & \SI{0.1}{\milli\meter} (4 mil) \\
        Minimum Via Diameter & \SI{0.45}{\milli\meter} \\
        Minimum Via Hole Diameter & \SI{0.3}{\milli\meter} \\
        Minimum BGA Pad Diameter & \SI{0.25}{\milli\meter} \\
        Maximum Board Size & \SI{100}{\milli\meter} × \SI{100}{\milli\meter} \\
        Number of Layers & 2, 4, or 6 layers \\
        \hline
    \end{tabular}
\end{table}

\begin{figure}[H]
    \centering
    \begin{minipage}{0.5\textwidth}
        \centering
        \includegraphics[width=\textwidth]{BioPal_render.png}
        \caption{Final PCB Render}
        \label{fig:final_pcb}
    \end{minipage}\hfill
    \begin{minipage}{0.5\textwidth}
        \centering
        \includegraphics[width=\textwidth]{BGA_Routing.png}
        \caption{\\OPA3S328 BGA Routing Solution}
        \label{fig:bga_routing}
    \end{minipage}
\end{figure}

\begin{figure}[H]
    \centering
    \begin{subfigure}[b]{0.75\textwidth}
        \centering
        \includegraphics[width=\textwidth]{BioPal_Front.png}
        \caption{PCB front layout}
        \label{fig:pcb_front}
    \end{subfigure}
    
    \vspace{1em}
    
    \begin{subfigure}[b]{0.75\textwidth}
        \centering
        \includegraphics[width=\textwidth]{BioPal_Back.png}
        \caption{PCB back layout}
        \label{fig:pcb_back}
    \end{subfigure}
    \caption{Final PCB Layout}
    \label{fig:pcb_layout}
\end{figure}

\chapter{Full Testing Results}
\makeatletter\@mkboth{}{Appendix}\makeatother
\label{appen:testing}


