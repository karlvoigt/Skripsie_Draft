\graphicspath{{conclusion/fig/}}

\chapter{Summary and Conclusion}
\label{chap:conclusion}
This project documents the development and testing of BioPal, a low-cost multiplexed impedance analyser aimed at point-of-care biosensing applications. The device integrates analogue frontend circuitry with embedded firmware and dual user interfaces to enable \ac{EIS} measurements on interdigitated electrodes without requiring specialised training or external equipment.

The analogue frontend implements voltage-controlled EIS across 1 Hz to 100 kHz using a 12-bit DAC for excitation, instrumentation amplifier-based voltage measurement, and transimpedance amplifier-based current measurement with programmable gain. A relay-based multiplexer enables sequential measurement of up to four \acp{IDE}. The complete system was integrated onto a custom PCB and housed in a battery-powered portable enclosure. Two user interfaces were implemented: an on-device LCD with buttons for standalone operation, and a web-based interface via Bluetooth Low Energy for enhanced usability.

The project achieved its core objectives. The analogue frontend successfully generates excitation signals and measures voltage and current responses across the target frequency range, enabling microcontroller-based EIS measurements (Objective 1). The multiplexer sequentially measures up to four IDEs (Objective 2). A complete PCB integrating all subsystems was designed and manufactured (Objective 3). Firmware implementing signal processing, FFT-based impedance calculation, and calibration correction was developed for both microcontrollers (Objective 4). Dual user interfaces enable standalone operation with qualitative risk assessment (Objective 5). The device operates from battery power with total component costs of approximately R2,000 (A full cost breakdown is shown in Table \ref{tab:cost} in Appendix \ref{appen:design}), well below the R4,500 target (Objective 6). Calibration against the PalmSens4 was completed using test cells and IDEs, with validation demonstrating sub-3\% error margins for PBS measurements and successful differentiation between varying concentrations of BSA protein binding (Objective 7).

However, several limitations constrain immediate deployment. 
\subsection{Limitations}
The inoperable LTC1069 AA filter limits accurate measurements to frequencies above 125 Hz due to aliasing artefacts, though this proved sufficient for the tested IDEs. This would be easy to replace given more time and would lead to even better accuracy and repeatability. The fixed-gain configuration improves stability but reduces flexibility for \acp{IDE} with different impedance characteristics. Validation was restricted to passive components, PBS solutions, and BSA protein. Testing with immobilised antibodies and disease biomarkers or clinical samples were outside the scope of this project. Despite these limitations, the BioPal demonstrates core measurement capabilities and a validated platform for further development toward practical \ac{POC} biosensing applications.

\subsection{Recommendations for Future Work}
Recommendations for future work towards achieving an accessible \ac{POC} biosensing system based on the BioPal platform inculde:
\begin{itemize}
    \item Enable gain settings to be configured via the user interface to accommodate \acp{IDE} with varying impedance ranges.
    \item Conduct tests using a variety of \acp{IDE} as well as immobilised antibodies and relevant disease biomarkers to evaluate real-world biosensing performance.
    \item Establish robust standardised measurement protocols to ensure consistent timing across tests.
    \item Perform user testing with healthcare workers to assess usability and workflow integration in real-world \ac{POC} settings.
\end{itemize}

These recommendations would build upon the validated BioPal platform to advance toward a practical, user-friendly impedance analyser deployed in point-of-care settings.